Dans notre étude, nous considérons un cylindre infiniment long en rotation uniforme. Il est intéressant de regarder les solutions analytiques qui ont été trouvées pour un cylindre de longueur finie. Dans le livre de référence \cite{greenspan1969theory}, ces solutions ont été trouvées par la méthode dites de Poincaré. Cette méthode a été étudiée en premier par Kelvin \cite{lord1880vibrations} et plus tard par Bjerknes et Solberg \cite{bjerknes1929zellulare}. On travaille ici en coordonnées généralisées $\bm{q}$ pour le champ de vitesse, avec les équations
\begin{equation}
    \nabla \cdot \bm{q} = 0 \,\ \,\ \text{;}\,\ \,\ \frac{\partial \bm{q}}{\partial t} + 2\mathbf{\Omega} \times \bm{q} = -\frac{1}{\rho}\nabla p'
\end{equation}
ou $p' = P - \frac{1}{2}\rho (s\Omega)^2 + \rho g z$, la pression réduite. On suppose des solutions en ondes-planes du type
\begin{equation}
    \bm{q} = \bm{Q}e^{i\omega t} \,\ \,\ \text{;} \,\ \,\ \frac{p'}{\rho} = \Phi e^{i\omega t}
\end{equation}
Pour étudier les ondes inertielles en coordonnées cylindriques, on exprime $\bm{q}$ en fonction de $\nabla \Phi$ et en prenant la divergence on obtient
\begin{equation}
    \frac{1}{s} \frac{\partial}{\partial s}s\frac{\partial\Phi}{\partial s} + \frac{1}{s^2} \frac{\partial^2 \Phi}{\partial \varphi^2} + \left( 1 - \frac{4\Omega^2}{\omega^2}\right)\frac{\partial^2 \Phi}{\partial z^2} = 0
\end{equation}
avec les conditions limites particulières suivantes
\begin{equation}
    \begin{cases}
    \frac{\partial \Phi}{\partial z} = 0 \,\ \text{en} \,\ z = 0,h \\
    i \lambda \frac{\partial \Phi}{\partial s} + \frac{2}{s} \frac{\partial \Phi}{\partial \varphi} = 0 \,\ \text{en} \,\ s = a
\end{cases}
\end{equation}
Les conditions aux bords réduisent le nombre de solutions à un ensemble infini discret. On peut résoudre ce problème par méthode de séparation des variables et on trouve que 
\begin{equation}
    \Phi_{nmk}(s,\varphi,z) = J_{\lvert m\rvert}(\xi_{nmk}s/a) \cos{(k\pi z)}\exp{(im\varphi)}
\end{equation}
pour $m = 0$ et $k,m = \pm 1, \pm 2,\dots$. La constante $a$ donne le rapport $\frac{rayon}{hauteur}$ du cylindre. Les fréquences propres $\lambda_{nmk}$ sont données par
\begin{equation}
    \lambda_{nmk} = \frac{\omega}{\Omega}=2 \left( 1 + \frac{\xi^2_{nmk}h^2}{k^2\pi^2 a^2}\right)^{-1/2}
\end{equation}
avec $\xi_{nmk}$ la n$^{ieme}$ solution positive de l'équation transcendante :
\begin{equation}
    \xi\frac{d}{d\xi}J_{\lvert m \rvert}(\xi) + m\left( 1 + \frac{\xi^2_{nmk} h^2}{k^2\pi^2 a^2}\right)^{1/2} J_{\lvert m \rvert}(\xi) = 0
\end{equation}
ou $J$ est une fonction de Bessel de première espèce. On peut écrire différent l'équation précédente grâce aux propriétés des fonctions de Bessel 
\begin{equation}
    \xi\left(-J_{\lvert m+1\rvert}(\xi) + \frac{m}{\xi}J_{\lvert m \rvert}(\xi)\right) + m\left( 1 + \frac{\xi^2 h^2}{k^2 \pi^2 a^2} \right)^{1/2}J_{\lvert m\rvert}(\xi) = 0
\end{equation}
Pour trouver les valeurs de $\xi$ qui satisfont l'équation transcendante, on utilise la méthode de \textit{Newton-Raphson} implémentée en langage \textit{python} et on trouve alors les racines pour différents paramètres.
\bigbreak
On voit que les valeurs propres ont tendance à être plus élevées pour $k>m$ et plus petites pour $k<m$. Si on force le fluide à une de ces fréquences, il y aura résonance. Comme expliqué dans les perspectives, en imposant une topographie initiale à la base du cylindre, on peut sélectionner différents modes de vibrations. Ce sont des valeurs importantes car, si on veut s'assurer de la qualité de notre simulation par la suite, on peut imposer des conditions à la base et au sommet du cylindre et, en reprenant les mêmes paramètres, essayer de retrouver les mêmes fréquences propres
